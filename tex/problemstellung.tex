\documentclass[12pt,a4paper]{scrartcl}
% scrartcl ist eine abgeleitete Artikel-Klasse im Koma-Skript
% zur Kontrolle des Umbruchs Klassenoption draft verwenden


% die folgenden Packete erlauben den Gebrauch von Umlauten und ß
% in der Latex Datei
\usepackage[utf8]{inputenc}
% \usepackage[latin1]{inputenc} %  Alternativ unter Windows
\usepackage[T1]{fontenc}
\usepackage[ngerman]{babel}

\usepackage{scrpage2}
\usepackage[pdftex]{graphicx}
\usepackage{latexsym}
\usepackage{amsmath,amssymb,amsthm}
\usepackage{latexsym}
\usepackage{amssymb}


% Abstand obere Blattkante zur Kopfzeile ist 2.54cm - 15mm
\setlength{\topmargin}{-15mm}


% Umgebungen für Definitionen, Sätze, usw.
% Es werden Sätze, Definitionen etc innerhalb einer Section mit
% 1.1, 1.2 etc durchnummeriert, ebenso die Gleichungen mit (1.1), (1.2) ..
\newtheorem{Satz}{Satz}[section]
\newtheorem{Definition}[Satz]{Definition} 
\newtheorem{Lemma}[Satz]{Lemma}	
\newtheorem{Beweis}{Beweis}	
                  
\numberwithin{equation}{section} 

% einige Abkuerzungen
\newcommand{\C}{\mathbb{C}} % komplexe
\newcommand{\K}{\mathbb{K}} % komplexe
\newcommand{\R}{\mathbb{R}} % reelle
\newcommand{\Q}{\mathbb{Q}} % rationale
\newcommand{\Z}{\mathbb{Z}} % ganze
\newcommand{\N}{\mathbb{N}} % natuerliche

\title{Integration im TrueSkill Verfahren}
\author{Johannes Loevenich}

\begin{document}

\maketitle

\begin{abstract}

\end{abstract}

\section{Problemstellung}
Angenommen es sei ein Parameter $\varTheta$ mit $P(\varTheta) = N(\varTheta; \mu;\varSigma)$ und eine Likelyhood Wahrscheinlichkeit $P(x|\varTheta)$ gegeben. 
Bezeichne die Likelyhood Wahrscheinlichkeit die Funktion $t_x(\varTheta)$, die nur vom Parameter $\varTheta$ abhängt. 
Dann ist die Wahrscheinlichkeit $P(\varTheta|x)$ nicht länger gaußverteilt, 

\begin{equation}
  P(\varTheta|x) = \frac{t_x(\varTheta)P(\varTheta)}{\int t_x(\varTheta^{'})P(\varTheta^{'})d\varTheta^{'}}
\end{equation}

Vom ADF wissen wir, dass wir diese Wahrscheinlichkeit mithilfe der Gaußverteilung $N(\varTheta,\mu_x^{'},\varSigma_x^{'})$ so approximieren können, dass die KL-Divergenz minimiert wird. 
Im Allgemeinen ergeben sich 

\begin{equation}
 \mu_x = \mu + \varSigma_x, \text{			} \varSigma_x^{'} = \varSigma - \varSigma(g_xg_x^T - 2G_x)\varSigma,
\end{equation}
 wobei der Vektor \textbf{$g_x$} und die Matrix $G_x$ durch  

 \begin{equation}
  g_x := \frac{\partial log(Z_x(\mu^{'},\varSigma^{'}))}{\partial \mu^{'}}, \text{			} G_x := \frac{\partial log(Z_x(\mu^{'},\varSigma^{'}))}{\partial \varSigma^{'}}
 \end{equation}
gegeben sind. 
 \end{document}
