\documentclass[12pt,a4paper]{scrartcl}
% scrartcl ist eine abgeleitete Artikel-Klasse im Koma-Skript
% zur Kontrolle des Umbruchs Klassenoption draft verwenden


% die folgenden Packete erlauben den Gebrauch von Umlauten und ß
% in der Latex Datei
\usepackage[utf8]{inputenc}
% \usepackage[latin1]{inputenc} %  Alternativ unter Windows
\usepackage[T1]{fontenc}
\usepackage[ngerman]{babel}
\usepackage{bbm}
\usepackage{scrpage2}
\usepackage[pdftex]{graphicx}
\usepackage{latexsym}
\usepackage{amsmath,amssymb,amsthm,amstext,amsfonts,mathrsfs}
\usepackage{latexsym}
\usepackage{amssymb}


% Abstand obere Blattkante zur Kopfzeile ist 2.54cm - 15mm
\setlength{\topmargin}{-15mm}

\parindent0pt 

% Umgebungen für Definitionen, Sätze, usw.
% Es werden Sätze, Definitionen etc innerhalb einer Section mit
% 1.1, 1.2 etc durchnummeriert, ebenso die Gleichungen mit (1.1), (1.2) ..
\newtheorem{Satz}{Satz}[section]
\newtheorem{Definition}[Satz]{Definition} 
\newtheorem{Lemma}[Satz]{Lemma}	
\newtheorem{Beweis}{Beweis}	
                  
\numberwithin{equation}{section} 

% einige Abkuerzungen
\newcommand{\C}{\mathbb{C}} % komplexe
\newcommand{\K}{\mathbb{K}} % komplexe
\newcommand{\R}{\mathbb{R}} % reelle
\newcommand{\Q}{\mathbb{Q}} % rationale
\newcommand{\Z}{\mathbb{Z}} % ganze
\newcommand{\N}{\mathbb{N}} % natuerliche

\title{Integration im TrueSkill Verfahren}
\author{Johannes Loevenich}

\begin{document}

\maketitle

\begin{abstract}

\end{abstract}

\section{Problemstellung}

\subsection{Gaussian Density Filtering}
Angenommen es sei ein Parameter $\varTheta$ mit $P(\varTheta) = N(\varTheta; \mu;\varSigma)$ und eine Likelyhood Wahrscheinlichkeit $P(x|\varTheta)$ gegeben. 
Bezeichne die Likelyhood Wahrscheinlichkeit die Funktion $t_x(\varTheta)$, die nur vom Parameter $\varTheta$ abhängt. 
Dann ist die Wahrscheinlichkeit $P(\varTheta|x)$ nicht zwingend länger gaußverteilt, 

\begin{equation}
  P(\varTheta|x) = \frac{t_x(\varTheta)P(\varTheta)}{\int t_x(\varTheta^{'})P(\varTheta^{'})d\varTheta^{'}}
\end{equation}

Vom ADF wissen wir, dass wir diese Wahrscheinlichkeit mithilfe der Gaußverteilung $N(\varTheta,\mu_x^{'},\varSigma_x^{'})$ so approximieren können, dass die KL-Divergenz minimiert wird. 
Im Allgemeinen ergeben sich 

\begin{equation}
 \mu_x = \mu + \varSigma_x, \text{			} \varSigma_x^{'} = \varSigma - \varSigma(g_xg_x^T - 2G_x)\varSigma,
\end{equation}
 wobei der Vektor \textbf{$g_x$} und die Matrix $G_x$ durch  

 \begin{equation}
  g_x := \frac{\partial log(Z_x(\mu^{'},\varSigma^{'}))}{\partial \mu^{'}}, \text{			} G_x := \frac{\partial log(Z_x(\mu^{'},\varSigma^{'}))}{\partial \varSigma^{'}}
 \end{equation}
gegeben sind. 

\subsection{EP Algorithmus für die Gaußverteilung}

Wie bereits im letzen Abschnitt nehmen wir an, dass wir einen Gaußverteilten Parameter $\Theta$ mit 
$?(\Theta) = \mathcal{N}(\Theta; ßmu, \varSigma)$ und einen Likelyhood $P(\mathbf{x}|\Theta)$ in m Faktoren gegeben haben,
sodass 

\begin{equation}
 P(\mathbf{x}|\Theta) = \prod_{i=1}^m t_{i,\mathbf{x}}(\Theta)
\end{equation}

Dann ist die Wahrscheinlichkeit $P(\varTheta|x)$ nicht zwingend länger gaußverteilt, 

\begin{equation}
  P(\varTheta|x) = \frac{t_x(\varTheta)P(\varTheta)}{\int t_x(\varTheta^{'})P(\varTheta^{'})d\varTheta^{'}}
\end{equation}

Wir können sogar nicht einmal mithilfe der KL-Divergenz die beste Approximation für den wirklichen Posterior finden, 
da wir die Ableitungen der aus Summen und Produkten bestehenden Normalisierungskonstante $Z_x$ nicht effektiv
berechnen können (\textit{Fluch der Dimension}). Wir wollen deshalb den Ansatz wählen die Faktoren 
$t_{i,x}$ nacheinander in den Posterior einbauen. Dieses Verfahren ist als \textit{Expectation Propagation} bekannt. 
Für Details möchten wir auf Kapitel () verweisen. \\

\textbf{Das Modell	} Wir nehmen an, dass der i-te Faktor des Likelyhoods Funktion einer niedrig-dimensionalen Projektion von $\Theta$ ist. 
Dann können wir die folgenden $m$ Funktionen $f_i$ statt den $m$ Faktoren $t_{i,x}$ verwenden:

\begin{equation}
 f_i(\Theta) := s_i exp(- \frac{1}{2} (\mathbf{A_i^T \Theta - \mu_i}) \Pi (\mathbf{A_i^T \Theta - \mu_i})).
\end{equation}

Dazu definieren wir $f_0(\Theta) := \mathcal{N}(\Theta; \mu; \varSigma)$. Die Approximation $P'(\Theta,x)$, des Posteriors, $P(\Theta,x)$ hat dann
die gleiche funktionale Form, 

\begin{equation}
 P'(\Theta,x) = \frac{ \prod_{i=0}^m f_i(\Theta)}{ \int \prod_{i=0}^m f_i(\Theta') d \Theta' } = \mathcal{N}(\Theta; \mu', \varSigma').
\end{equation}

Wir möchten an dieser Stelle bemerken, dass aufgrund der Projektion die Funktion 
$Z_i := \int t_{ix}(\mathbf{\Theta}) \mathcal{N}(\mathbf{\Theta; \mu, \varSigma}) d \mathbf{\Theta}$, der Vektor
$\mathbf{g_i} := \partial \log(Z_i(\mathbf{\mu', \varSigma'}))/\partial \mu'$ und die Matrix
$\mathbf{G_i} := \partial \log(Z_i(\mathbf{\mu',\varSigma'}))/\partial \mathbf{\varSigma'}$ die folgende Form haben,

\begin{equation}
 \begin{split}
 Z_i(\mathbf{\mu, \varSigma}) &= \int h(y) \mathcal{N}(y;\mathbf{A_i^T \mu, A_i^T \varSigma A_i})dy \\
 \mathbf{g_i(\mu, \varSigma)} &= \mathbf{A_i[\alpha_i(A_i^T \mu, A_i^T \varSigma A_i)]}, \\
 \mathbf{g_i(\mu, \varSigma) g_i^T(\mu, \varSigma)} - 2 \mathbf{G_i(\mu,\varSigma)} &= \mathbf{A_i[\Gamma_i(A_i^T \mu, A_i^T \varSigma A_i)] A_i^T}.\\
 \end{split}
 \end{equation}

Dabei ist $\mathbf{\alpha_i}$ eine vektorwertige und $\Gamma_i$ eine matrixwertige Funktion für den i-ten Faktor.\\

\textbf{Der Algorithmus} Wir nehmen zu Beginn an, dass $\mathbf{\mu_i = 0, \Pi_i = 0} $ und $s_i = 1$. 
Damit folgt, dass $f_i$ die konstante 1-Funktion ist und damit $\mathbf{\hat \mu = \mu , \hat \varSigma = \varSigma} \text{ und } \hat Z_x = 1 $.
Die Idee des \textit{EP-Algorithmus} ist es nacheinander eine Faktor $t_j$ zu wählen, um somit die Approximation
$f_j$ mittels Korrekturen der Parameter $s_j, \mathbf{m_j} \text{ und } \mathbf{\Pi_j}$ zu verbessern. 
Der resultierende Algorithmus \textit{EP-Algorithmus} ist dann: 

\begin{enumerate}
 \item Berechne die Parameter $ \mathbf{\mu_{\setminus j}}, \mathbf{\varSigma{\setminus j}} $ der Gaußapproximation des Posteriors ohne
  den Faktor $t_j$ aber mit allen anderen Faktoren, 
  \begin{equation}
   P_{\setminus j}(\mathbf{\Theta|x}) = \mathcal{N}(\mathbf{\Theta; \mu_{ \setminus j} , \varSigma_{\setminus j}}) = \frac{\prod_{i \neq j}{f_i(\mathbf{\Theta})}}{ \int \prod_{i \neq j}{f_i(\mathbf{\tilde \Theta})} d \mathbf{\tilde \Theta} }
  \end{equation}

  \item Benutze die \textit{Gaussian Density Filtering} Approximation, wie in () vorgestellt um die neue 
  Approximation des Posteriors zu erhalten, $\hat P(\mathbf{\Theta|x})$, mit
  \begin{equation}
   \hat P(\mathbf{\Theta|x}) = \mathcal{N}(\mathbf{\Theta, \hat \mu, \hat \varSigma}) = \frac{ t_{j,x}(\mathbf{\Theta}) \cdot \mathcal{N}(\mathbf{\Theta; \mu_{\setminus j}, \varSigma_{\setminus j}})}{\int t_{j,x}(\mathbf{ \tilde \Theta}) \cdot \mathcal{N}(\mathbf{\tilde \Theta; \mu_{ \setminus j}, \varSigma_{\setminus\ j}} d \tilde \Theta ) }.
  \end{equation}
  Dies kann nach Annahme effizient für jeden einzelnen Faktor berechnet werden. 
  \item Akutalisiere die Parameter $s_j, \mathbf{\mu_j \text{ und } \Pi_j}$ des Faktors $f_j$, s.d.
  
  \begin{equation}
   \mathcal{N}(\mathbf{\Theta, \hat \mu, \hat \varSigma}) = \frac{ f_j(\mathbf{\Theta}) \cdot \mathcal{N}(\mathbf{\Theta; \mu_{\setminus j}, \varSigma_{\setminus j}})}{\int f_j(\mathbf{ \tilde \Theta}) \cdot \mathcal{N}(\mathbf{\tilde \Theta; \mu_{ \setminus j}, \varSigma_{\setminus\ j}} d \tilde \Theta ) }
  \end{equation}
   und zur gleichen Zeit
   \begin{equation}
    \int f_j(\mathbf{ \tilde \Theta}) \cdot \mathcal{N}(\mathbf{\tilde \Theta; \mu_{ \setminus j}, \varSigma_{\setminus\ j}} d \tilde \Theta ) = \int t_{j,x}(\mathbf{ \tilde \Theta}) \cdot \mathcal{N}(\mathbf{\tilde \Theta; \mu_{ \setminus j}, \varSigma_{\setminus\ j}} d \tilde \Theta ) 
   \end{equation}
   gilt. () Ist nicht ausreichend, um alle Parameter zu aktualisieren, da $s_j$ im Zähler und Nenner der linken Seite
   der Gleichung auftaucht. 
\end{enumerate}

\textbf{Wichtige Beziehungen } Um die Gleichungen der Aktualisierungen für die j-te Funktion zu erhalten benutzen wir
(1.3) in (3.4)

\begin{equation}
 \mathbf{\hat \varSigma \hat \mu = A_j \Pi_j \mu_j + \varSigma^{-1}_{\setminus j} \mu_{\setminus j}}, \text{	} \mathbf{\hat \varSigma^{-1} = A_j \Pi_j A_j^T + \varSigma_{\setminus j}^{-1}}. 
\end{equation}

Mit der Eigenschaft aus (1.4) gilt außerdem 

\begin{equation}
 \int f_j(\mathbf{ \tilde \Theta}) \cdot \mathcal{N}(\mathbf{\tilde \Theta; \mu_{ \setminus j}, \varSigma_{\setminus\ j}}) d \tilde \Theta = s_j (2 \pi)^{\frac{n}{2}} |\mathbf{\Pi_j}|^{-\frac{1}{2}} \mathcal{N}(\mathbf{\mu_j; A_j^T \mu_{\setminus j}, \Pi_j^{-1} + A_j^T \varSigma_{\setminus j} A_j})
\end{equation}

Im Weiteren werden wir effiziente und numerisch stabile Gleichungen zur Aktualisierung der Parameter aufstellen. 
Dabei benutzen wir die Kurznotationen

\begin{equation}
 \begin{split}
 \mathbf{U_j} &:= \mathbf{\hat \varSigma A_j}, \text{	}, \mathbf{C_j := A_j^T U_j}, \text{	} \mathbf{m_j := A_j^T \hat \mu}, \text{	} \mathbf{D_j:= C_j \Pi_j ,} \\
 \mathbf{ E_j}  &:= \mathbf{(I- D_j)^{-1}} , \text{		} \mathbf{F_j := (I-D_j^T)^{-1} = I + \Pi_j E_j C_j  } . \\
 \end{split}
 \end{equation}

Der Ausdruck für $\mathbf{F_j}$ folgt hierbei mithilfe der \textit{Woodbury Formel}. Ein vollständiger Algorithmus ist in Algorithmus 1 angegeben. 
Für den Rang 1, d.h. $\mathbf{A_i = a_i}$ lässt sich der Algorithmus ohne die Berechnung der Inversen schreiben. Diese Modifikation berücksichtigt
Algorithmus 2. \\

\textbf{Eliminationsgleichungen	} Um die $j-$te Funktion zu entfernen benutzen wir die \textit{Woodbury Formel} und nutzen die Symmetrie
von $\hat \varSigma$. Damit erhalten wir 

\begin{equation}
 \begin{split}
  \mathbf{\varSigma_{\setminus j}} &= \mathbf{( \hat \varSigma^{-1} - A_j \Pi_j A_j^T)} \\
  &= \mathbf{\hat \varSigma + (\hat \varSigma A_j) \Pi_j (I - A_j^T \hat \varSigma A_j \Pi_j)^{-1} (A_j^T \hat \varSigma)}, \\
  \mu_{\setminus j} &= \mathbf{\varSigma_{\setminus j} (\hat \varSigma^{-1} \hat \mu - A_j \Pi_j \mu_j)} \\
  &= \mathbf{\hat \mu \varSigma_{\setminus j} A_j \Pi_j (A_j^T \hat \mu - \mu_j) }\\
  &= \mathbf{\hat \mu + ( \hat \varSigma A_j \Pi_j )( I - A_j^T \hat \varSigma A_j \Pi_j )^ {-1} (A_j^T \hat \mu - \mu_j)} \\
 \end{split}
\end{equation}


ALGORITHMUS 1 EINFüGEN!!! \\


(3.11) folgt durch einsetzen von $\mathbf{A_j \Pi_j A_j^T + \varSigma_{\setminus j}^{-1}}$ für $\hat \varSigma^{-1}$ und (3.12) aus (3.10) mittels Multiplikation
von $\mathbf{A_j}$ und Einsetzen in (3.11). Damit und mit (3.8) können die Gleichungen zum Entfernen der $j-$ten
Funktion umgeschrieben werden,

\begin{equation}
 \mathbf{\mu_{\setminus j} = \hat \mu U_j \Pi_j E_j(m_j - \mu_j), \varSigma_{\setminus j} = \hat \varSigma U_j \Pi_j E_j U_j^T.} \\ 
\end{equation}

Mithilfe der Kurznotationen in () und () lassen sich die Ausdrücke als 

\begin{equation}
 \mathbf{\Phi_j := A_j^T \mu_{\setminus j} = m_j + D_j E_j(m_j - \mu_j), \text{		} \Psi_j = A_j^T \varSigma_{\setminus j}A_j = E_j C_j}
\end{equation}

schreiben. 

\textbf{GDF Update Gleichungen	} In Anlehnung an Kapitel () sind die Update Gleichungen nach der Eleminierung des $j-$ten Faktors
leicht einzusehen 

\begin{equation}
 \begin{split}
  \mathbf{\hat \mu_{new}} &=  \mathbf{\hat \mu + U_j[\Pi_j E_j(m_j - \mu_j) + F_j \cdot \alpha_j(\Phi_j, \Psi_j)]} \\
  \mathbf{\hat \varSigma_{new}} &= \mathbf{ \hat \varSigma + U_j[\Pi_j - F_j \cdot \Gamma_j(\Phi_j, \Psi_j)] E_j U_j^T}. \\ 
 \end{split}
\end{equation}

\textbf{Faktor Update Gleichungen} Wir können die \textit{Woodbury Formel} auf (3.14) anwenden, um eine Inverse der neuen Kovarianzmatrix
$ \hat \varSigma_{new}$ zu berechnen

\begin{equation}
  \hat \varSigma_{new}^{-1} = \mathbf{ \varSigma_{\setminus j}^{-1} + A_j(\Gamma_j^{-1}(\Phi_j, \Psi_j)- \Psi_j)^{-1} A_j^T}.
\end{equation}

Ebenso können wir $ \hat \varSigma_{new}^{-1} \hat \mu_{new}$ ausdrücken, 

\begin{equation}
 \mathbf{ \hat \varSigma_{new}^{-1} \hat \mu_{new} = \varSigma_{\setminus j}^{-1} \mu_{\setminus j} + A_j[[\Gamma_j^{-1}(\Phi_j, \Psi_j) - \Psi_j]^{-1} [\Gamma_j^{-1}(\Phi_j, \Psi_j) \cdot \alpha_j(\Phi_j, \Psi_j) + \Phi_j]]. \\ }
\end{equation}

Nun benutzen wir (3.6) und (3.7), um die Update Gleichungen für den $j-$ten Faktor aufschreiben zu können, 

\begin{equation}
 \begin{split}
  \mathbf{\Pi_{j,new}} &= \mathbf{(\Gamma_j^{-1}(\Phi_j, \Psi_j) - \Psi_j)^{-1}}, \\
  \mathbf{\mu_{j, new} } &= \mathbf{\Gamma_j^{-1}(\Phi_j, \Psi_j) \cdot \alpha_j(\Phi_j,\Psi_j) + \Phi_j} \\
  s_{j, new} &= Z_j(\mathbf{\Phi_j, \Psi_j}) \cdot \mathbf{|I - \Gamma_j(\Phi_j, \Psi_j) \cdot \Psi_j|^{-\frac{1}{2}} exp( \frac{1}{2} \alpha_j^T(\Phi_j, \Psi_j) \cdot \Gamma_j^{-1}(\Phi_j, \Psi_j) \cdot \alpha_j(\Phi_j, \Psi_j))}. \\
 \end{split}
\end{equation}

\textbf{Normalisierungskonstante	} Die Normalisierung $ \hat Z_x = \hat P(x)$ der Approximation von $\mathbf{P(x)}$ kann aus (3.1) einfach ermittelt
 werden indem $\mathbf{\Theta = 0}$ gesetzt wird, 
 
\begin{equation}
 \hat Z_x = (\prod_{i = 1}^m s_i) \cdot \sqrt{ \frac{| \hat \varSigma |}{|\varSigma|}} \cdot exp(- \frac{1}{2}( \sum_{i = 1}^m \mathbf{\mu_i^T \Pi_i \mu_i + \mu^T \varSigma^{-1} \mu - \hat \mu^T \hat \varSigma^{-1} \hat \mu} )).
\end{equation}

\subsection{Multidimensional korrigiert und abgeschnittene Gaußverteilung}
Wir nennen $x$ doppelseitig korrigiert und abgeschnitten gaußverteilt, wenn $x \sim \mathcal{R}(x; \mu, \sigma^2, l,u)$ und dies bedeutet, dass die Dichte von $\mathbf{x}$ durch 

\begin{equation}
 \begin{split}
 \mathcal{R}(x; \mu, \sigma^2, l,u) &= \mathbbm{1}_{x \in (l,u)} \cdot \frac{ \mathcal{N}(x; \mu, \sigma^2) }{ \Phi(u; \mu, \sigma^2) - \Phi(l; \mu, \sigma^2)} \\
 &= \mathbbm{1}_{x \in (l,u)} \cdot \frac{ \mathcal{N}(\frac{x- \mu}{\sigma}) }{\sigma \cdot (\Phi(\frac{u-\mu}{\sigma} - \Phi(\frac{l-\mu}{\sigma})))} \\
 \end{split}
 \end{equation}

gegeben ist. Wir schreiben $\mathcal{R}(x; \mu, \sigma^2, l)$ für $\lim_{u \rightarrow + \infty} \mathcal{R}(x; \mu, \sigma^2, l,u)$. Oft wird diese Verteilung einfach
abgeschnittene Gaußverteilung genannt. Wir möchten bemerken, dass die Klasse der abgeschnittenen Gaußverteilungen die Familie der Gaußverteilung als
Spezialfall enthält, 

\begin{equation}
 \lim_{l \rightarrow - \infty} \mathcal{R}(x; \mu, \sigma^2, l) = \mathcal{N}(x;\mu, \sigma^2).
\end{equation}

Es gelten die folgenden Eigenschaften für die Erwartung und die Varianz beidseitig abgeschnittener Gaußverteilungen. 

\begin{Satz}
 Die Erwartung und Varianz der abgeschnittenen Gaußverteilung haben die Form
 
 \begin{equation}
 \begin{split}
  \left< x \right>_{x \sim \mathcal{R}} &= \mu + \sigma \cdot v\left(\frac{\mu}{\sigma}, \frac{l}{\sigma}, \frac{u}{\sigma}\right), \\
  \left< x^2 \right>_{x \sim \mathcal{R}} - ( \left< x \right>_{x \sim \mathcal{R}})^2 &= \sigma^2 \cdot \left( 1 - w \left( \frac{\mu}{\sigma}, \frac{l}{\sigma}, \frac{\mu}{\sigma}\right) \right), \\
 \end{split}
 \end{equation}
\end{Satz}

 dabei sind die Funktionen $v$ und $w$ gegeben durch
 
 \begin{equation}
  \begin{split}
    v(t,l,u) &:= \frac{ \mathcal{N}(l-t) - \mathcal{N}(u-t) }{ \Phi(u-t) - \Phi(l-t)}, \\
    w(t,l,u) &:= v^2(t,l,u) + \frac{(u-t) \cdot \mathcal{N}(u-t) - (l-t) \cdot \mathcal{N}(l-t)}{\Phi(u-t) - \Phi(l-t)}. \\
  \end{split}
 \end{equation}

 \begin{Beweis}
  Um das erste und zweite Moment der abgeschnittenen Gaußverteilung berechnen zu können nutzen wir die Symmetrie
   der Funktion $\Phi$ in ihren ersten beiden Argumenten aus, 
   
   \begin{equation}
    \Phi \left(\frac{u - \mu}{\sigma} \right) \Phi(u; \mu,\sigma^2) = 1 - \Phi(\mu; u, \sigma^2).
   \end{equation}

   Die exakte Gleichung für das erste Moment erhalten wir dann, indem wir die erste Ableitung von 
   $\Phi \left( \frac{u- \mu}{\sigma} \right) - \Phi \left( \frac{l- \mu}{\sigma} \right)$ nach $\mu$ bilden. 
   Das zweite Moment folgt ebenso mittels der zweiten Ableitung des Ausdrucks nach $\mu$. 
 \end{Beweis}

\textbf{Erstes Moment	} Mittels der erwähnten Symmetrie erhalten wir 

\begin{equation}
 \frac{ \partial \left( \Phi \left( \frac{u - \mu}{\sigma}\right) - \Phi \left( \frac{l - \mu}{\sigma}\right) \right) }{ \partial \mu} = \int_l^u \frac{ \partial \mathcal{N}(x; \mu, \sigma^2)}{\partial \mu} dx = \sigma^{-2} \left< x -\mu \right>_{x \sim \mathcal{R}}
\end{equation}

und zur gleichen Zeit
\begin{equation}
 \begin{split}
 \frac{ \partial \left( \Phi \left( \frac{u - \mu}{\sigma}\right) - \Phi \left( \frac{l - \mu}{\sigma}\right) \right) }{ \partial \mu} &= \frac{ \partial (\Phi(ßmu; l, \sigma^2) - \Phi(\mu;u,\sigma^2))}{ \partial \mu} \\ 
 &= \mathcal{N}(\mu; l, \sigma^2) - \mathcal{N}(\mu; u, \sigma^2). \\
 \end{split}
 \end{equation}

Da der erste und der zweite Ausdruck die Gleichheit erfüllen müssen, schließen wir, dass 

\begin{equation}
 \begin{split}
   \left < x - \mu \right>_{x \sim \mathcal{R}} &= \sigma^2 \cdot \frac{ \mathcal{N}(\mu, l, \sigma^2) - \mathcal{N}(\mu; u, \sigma^2) }{ \Phi(u;\mu,\sigma^2) - \Phi(l;\mu,\sigma^2)} \\
   \left< x \right>_{x \sim \mathcal{R}} &= \mu + \sigma^2 \cdot \frac{ \mathcal{N}(\mu, l, \sigma^2) - \mathcal{N}(\mu; u, \sigma^2) }{ \Phi(u;\mu,\sigma^2) - \Phi(l;\mu,\sigma^2)}. \\
 \end{split}
\end{equation}

\textbf{Zweites Moment	} Um das zweite Moment zu berechnen nutzen wir wiederum die Symmetrie in Ausdruck (). Damit erhalten wir

\begin{equation}
\begin{split}
 \frac{\partial^2 \left (  \Phi \left ( \frac{u- \mu}{\sigma} \right ) - \Phi \left ( \frac{l- \mu}{\sigma} \right ) \right)}{\partial \mu^2} &= \int_l^u \frac{\partial \left ( \frac{x - \mu}{\sigma^2} \right ) \cdot \mathcal{N}(x; \mu, \sigma^2 )}{\partial \mu} dx \\
 &= \sigma^{-4} \left < (x - \mu)^2 - \sigma^2 \right >_{x \sim \mathcal{R}} . \\
\end{split}
\end{equation}

Gleichzeitig gilt

\begin{equation}
\begin{split}
\frac{ \partial^2 \left ( \Phi \left ( \frac{u- \mu}{\sigma} \right )  - \Phi \left ( \frac{u- \mu}{\sigma} \right )  \right )}{ \partial \mu^2} &= \frac{ \partial \left ( \mathcal{N}(\mu; l, \sigma^2) - \mathcal{N}(\mu;u, \sigma^2) \right ) }{ \partial \mu} \\
&= \sigma^{-2} \left [ (\mu - u) \mathcal{N}(\mu;u, \sigma^2) \right ] . \\
\end{split}
\end{equation}

Fügen wir die beiden Ausdrücke zusammen, so erhalten wir 

\begin{equation}
\begin{split}
 \left < x^2 - 2x \mu + \mu^2 - \sigma^2 \right >_{x \sim \mathcal{R}} = \sigma^2 \cdot \frac{(\mu - u) \cdot \mathcal{N}(\mu;u,\sigma^2) - (\mu - l) \cdot \mathcal{N}(\mu;l,\sigma^2)}{\Phi(u; \mu, \sigma^2) - \Phi(l; \mu, \sigma^2)} \\
 \left < x^2 \right >_{x \sim R} = \mu^2 + \sigma^2 \cdot \frac{(\mu + u) \cdot \mathcal{N}(\mu;u,\sigma^2) - (\mu + l) \cdot \mathcal{N}(\mu;l,\sigma^2)}{\Phi(u; \mu, \sigma^2) - \Phi(l; \mu, \sigma^2)} \\
 \end{split}
 \end{equation}
 
\textbf{Varianz	} Die Varianz einer abgeschnittenen Gaußverteilung ergibt sich aus der Differenz des zweiten und dem Quadrat des ersten Momentes, 

\begin{equation}
 \sigma^2 \left ( 1 - v^2 \left( \frac{\mu}{\sigma}, \frac{l}{\sigma}, \frac{\mu}{\sigma} \right) + \frac{(\mu - u) \cdot \mathcal{N}(\mu; u, \sigma^2) - (\mu - l) \cdot \mathcal{N}(\mu;l,\sigma^2)}{\Phi(u;\mu, \sigma^2) - \Phi(l; \mu, \sigma^2)}\right ).
\end{equation}

Diese Funktion bezeichnen wir mit $w$.

Im Grenzwert von $ u \rightarrow + \infty$ vereinfachen sich die Funktionen $v$ und $w$ zu

\begin{equation}
 \begin{split}
  v(t,l) &:= \lim_{u \rightarrow + \infty} v(t,l,u) = \frac{\mathcal{N}(t-l)}{\Phi(t-l)}, \\
  w(t,l) &:= \lim_{u \rightarrow + \infty} w(t,l,u) = v(t,l) \cdot (v(t,l) + t(t-l)).
 \end{split}
\end{equation}

Wir möchten dabei bemerken, dass die Funktion $w$ jederzeit durch das Intervall $[0,1]$ beschränkt ist. 
Die Funktion $v$ wächst wie $l-t$ für $t<l$ und fällt schnell zu Null für $t > l$. Außerdem ist die Funktion
$w$ eine glatte Approximation der Indikatorfunktion $\mathbbm{I}_{t \leq l}$.

\subsection{Mehrdimensional abgeschnittene Gaußverteilung}

Wir nennen $\mathbf{x}$ mehrdimensional abgeschitten und Gaußverteilt, falls 
$\mathbf{x \sim \mathcal{R}(x;\mu,\varSigma,l,u)} $, d.h., dass die Dichte von $\mathbf{x}$ die Form

\begin{equation}
 \mathcal{R}(\mathbf{x; \mu, \varSigma, l,u}) = \frac{ \mathbbm{I}_{\mathbf{l<x<u}} \cdot \mathcal{N}(\mathbf{x;\mu,\varSigma})}{ \mathbbm{I}_{\mathbf{l< \tilde x <u}} \cdot \mathcal{N}(\mathbf{\tilde x;\mu,\varSigma})}
\end{equation}

Bis jetzt wurden noch keine effizienten analytischen Ausdrücke für die Normalisierungskonstante und die Momente dieser
Verteilung gefunden. Wir können aber den Gausschen EP Algorithmus benutzen, um die Erwartung, Kovarianz und die Normalisierungskonstante
dieser Verteilung zu berechnen. Dazu schreiben wir den obigen Ausdruck zunächst um: 

\begin{equation}
 \mathcal{R}(\mathbf{x;\mu, \varSigma, l, u}) = \frac{ \prod_{i = 1}^n \mathbbm{I}_{l_i < x_i < u_i} \cdot \mathcal{N}(\mathbf{x;\mu,\varSigma})}{\int \prod_{i=1}^n \mathbbm{I}_{l_i< \tilde x_i < u_i } \cdot \mathcal{N}(\mathbf{\tilde x; \mu, \varSigma}) d \mathbf{\tilde x}}. 
\end{equation}

Nun müssen wir die exakten Gleichungen für $Z_i(\mathbf{\mu, \varSigma}), \alpha_i(\mathbf{\mu, \varSigma})$ und $\gamma_i(\mathbf{\mu,\varSigma})$ berechnen.
Dabei gilt

\begin{equation}
 \begin{split}
  \frac{\partial \log{Z_i}}{\partial \mathbf{\mu}} &= \alpha_i(\mathbf{\mu, \varSigma}) \cdot \mathbf{a_i} \\
  \left [ \frac{\partial \log{Z_i} }{\partial \mathbf{\mu}} \right ] \left [ \frac{\partial \log{Z_i}}{\partial \mathbf{\mu}}\right ]^T - 2 \cdot \frac{\partial \log(Z_i)}{\partial \mathbf{\varSigma}} &= \gamma_i(\mathbf{\mu, \varSigma}) \cdot \mathbf{a_i a_i^T}.
 \end{split}
\end{equation}

Wir nehmen im Folgenden an, dass die individuellen Faktoren die Form 

\begin{equation}
 t_i(\mathbf{x}) = \mathbbm{I}_{l_i < \mathbf{a_i^T x} < u_i}
\end{equation}

haben. Die exakte Form folgt durch setzen von $\mathbf{a_i = e_i}$. Um die Ausdrücke etwas zu kürzen 
benutzen wir die Notationen 

\begin{equation}
 \phi_i := \mathbf{a_i^T \mu} \text{	und	} \psi_i := \mathbf{a_i^T \varSigma a_i}.
\end{equation}

Mit () folgt für $Z_i(\mathbf{\mu,\varSigma})$, 

\begin{equation}
 Z_i(\mathbf{\mu, \varSigma}) = \int t_i(\mathbf{x}) \mathcal{N}(\mathbf{x; \mu, \varSigma}) d \mathbf{x} = \int \mathbbm{I}_{l_i < y < u_i} \mathcal{N}(y; \phi_i, \psi_i) dy = \Phi(\frac{u_i - \phi_i}{\sqrt{\psi_i}}) - \Phi(\frac{l_i - \phi_i}{\sqrt{\psi_i}}).
\end{equation}

Wir beginnen mit der Ableitung von $\log Z_i(\mathbf{\mu, \varSigma})$ nach $\mathbf{\mu}$. Mit der Kettenregel erhalten wir

\begin{equation}
 \frac{\partial \log Z_i(\mathbf{\mu, \varSigma})}{\partial \mathbf{\mu}} = \frac{1}{Z_i(\mathbf{\mu, \varSigma})} \cdot \frac{\partial \Phi(\frac{u_i - \phi_i}{\sqrt{\psi_i}}) - \Phi(\frac{l_i - \phi_i}{\sqrt{\psi_i}})}{\partial \phi_i} \cdot \frac{\partial \phi_i}{\mathbf{\mu}} = \frac{1}{\sqrt{\psi_i}} \cdot \frac{\mathcal{N}(\frac{l_i - \phi_i}{\sqrt{\psi_i}}) - \mathcal{N}(\frac{u_i - \phi_i}{\sqrt{\psi_i}})}{\Phi(\frac{u_i - \phi_i}{\sqrt{\psi_i}}) - \Phi(\frac{l_i - \phi_i}{\sqrt{\psi_i}})} \cdot \mathbf{a_i}.
\end{equation}

Benutzen wir die in () definierte Funktion $v$, so können wir $\alpha_i(\phi, \psi_i)$ als

\begin{equation}
 \alpha_i(\phi_i, \psi_i) = \frac{1}{\sqrt{\psi_i}} \cdot v \left ( \frac{\phi_i}{\sqrt{\psi_i}}, \frac{l_i}{\sqrt{\psi_i}},\frac{u_i}{\sqrt{\psi_i}} \right )
\end{equation}

ausdrücken. 

Um schlussendlich $gamma_i$ zu berechnen entwickeln wir die partielle Ableitung von $\log{Z_i(\mathbf{\mu, \varSigma})}$ nach $\mathbf{\varSigma}$ mit Hilfe der Kettenregel
und erhalten

\begin{equation}
 \gamma_i(\phi_i, \psi_i) = \frac{1}{\psi_i} \cdot w \left ( \frac{\phi_i}{\sqrt{\psi_i}}, \frac{l_i}{\sqrt{\psi_i}},\frac{u_i}{\sqrt{\psi_i}} \right ).
\end{equation}

Den daraus resultierenden Algorithmus sehen wir in Algortihmus 3. 

\subsection{Transformationstechniken und Genz Algorithmus}
Im vorherigen Abschnitt haben wir gesehen, dass Integrale der Form

\begin{equation}
 F(\mathbf{\alpha},\mathbf{\beta}) = (2\pi)^{-\frac{n}{2}} |\Sigma|^{-\frac{1}{2}} \int_{\alpha_1}^{\beta_1} ... \int_{\alpha_n}^{\beta_n} exp(-\frac{1}{2}(\mathbf{x}-\mathbf{\mu})^T \mathbf{\Sigma}^{-1}(\mathbf{x}-\mathbf{\mu)}) g(\mathbf{x}) dx
\end{equation}

gelöst werden müssen. 
Auch wenn die in vorgestellten Integrationsmethoden in manchen Fällen zu einer Lösung dieses Integrals führen würden, wird die Konvergenz dieser Verfahren durch die hier vorgestellten
Verfahren deutlich verbessert. Das Hauptaugenmerk soll dabei auf der Transformation auf den Einheitswürfel $(0,1)^d$ liegen, um das so erhaltene Integral dann mithilfe von verschiedenen
Integrationsmethoden leicht lösen zu können. 

In einem ersten Schritt kann der Einfluss des Erwartungswertes $\mathbf{\mu}$ durch die geschickte Substitution $\mathbf{y} = \mathbf{x} - \mathbf{\mu}$ eliminiert werden,

\begin{equation}
  F(\mathbf{\alpha},\mathbf{\beta}) = (2\pi)^{-\frac{n}{2}} |\Sigma|^{-\frac{1}{2}} \int_{\alpha_1 -\mu_1}^{\beta_1-\mu_1} ... \int_{\alpha_n - \mu_n}^{\beta_n - \mu_n} exp(-\frac{1}{2} \mathbf{y}^T \mathbf{\Sigma}^{-1}\mathbf{y}) g(\mathbf{y} + \mathbf{\mu}) dx
\end{equation}

Sei nun $\mathbf{\Sigma = L L^T}$ die Cholesky-Zerlegung der Kovarianzmatrix $\mathbf{\Sigma}$ und $\mathbf{L}$ dabei untere Dreiecksmatrix, dann schreibe $\mathbf{y = Lz}$. Da 
$\mathbf{y}$ untere Dreicksmatrix ist, kann $\mathbf{z}$ iterativ durch 

\begin{equation}
 z_i = \frac{y_i - \sum_{j=1}^{i-1} L_{i,j}z_j}{L_{i,i}}
\end{equation}

bestimmt werden. Durch die positiv Definitheit von $\Sigma$ gilt stets $L_{i,i} > 0$ und $z_i$ ist strikt monoton steigende Funktion in Abhängigkeit von $y_i$.
Außerdem gilt $\mathbf{y^T\Sigma^{-1}y}$ \\ $ \mathbf{= z^TL^T(LL^T)^{-1}Lz} = \mathbf{z^Tz} $ und 

\begin{equation}
 d\mathbf{y} = |\mathbf{L}|d \mathbf{z} = |\mathbf{\Sigma}^{\frac{1}{2}}| d \mathbf{z}.
\end{equation}

Mit diesen zwei Eigenschaften lässt sich (1.5) umformen zu 

\begin{equation}
  F(\mathbf{\alpha},\mathbf{\beta}) = (2\pi)^{-\frac{n}{2}} |\Sigma|^{-\frac{1}{2}} \int_{\alpha_1'}^{\beta_1'} \mathcal{N}(z_1) \int_{\alpha_n'(z_1)}^{\beta_n'(z_1)} \mathcal{N}(z_2) \text{ ...} \int_{\alpha_n'(z_1,...,z_n)}^{\beta_n'(z_1,...,z_n)} \mathcal{N}(z_n) g(\mathbf{y} + \mathbf{\mu}) dx
\end{equation}

, wobei Funktionen $\alpha'$ und $\beta'$ durch 

\begin{equation}
  \begin{split}
    \alpha_i'(z_1,...,z_{i-1})  &=  \frac{\alpha_i - \mu_i - \sum_{j=1}^{i-1} L_{i,j}z_j}{L_{i,i}} \\
    \beta_i'(z_1,...,z_{i-1})   &=  \frac{\beta_i - \mu_i - \sum_{j=1}^{i-1} L_{i,j}z_j}{L_{i,i}} 
  \end{split}
\end{equation}

Als nächster Schritt wird eine Transformation der einzelnen Koordination mit der inversen Normalverteilung $\Phi^{-1}(z_i) = v_i$ durchgeführt. Damit ergibt sich mit

\begin{equation}
\begin{split}
 \alpha_i''(v_1,...,v_{i-1})&=  \Phi (\frac{\alpha_i - \mu_i - \sum_{j=1}^{i-1} L_{i,j}\Phi^{-1}(v_j)}{L_{i,i}} )\\
 \beta_i''(v_1,...,v_{i-1})&=  \Phi(\frac{\beta_i - \mu_i - \sum_{j=1}^{i-1} L_{i,j}\Phi^{-1}(v_j)}{L_{i,i}} )
 \end{split}
\end{equation}

für $F(\alpha,\beta)$ die Darstellung

\begin{equation}
  F(\mathbf{\alpha},\mathbf{\beta}) = \int_{\alpha_1''}^{\beta_1''} \int_{\alpha_2'(v_1)}^{\beta_2'(v_1)} \text{ ...} \int_{\alpha_n'(v_1,...,v_n)}^{\beta_n'(v_1,...,v_n)} g(\mathbf{L}[\Phi^{-1}(v_1),...,\Phi^{-1}(v_n)] + \mathbf{\mu}) d\mathbf{v}
\end{equation}

Zum Schluss führt eine Anwendung des Transformationssatz mit der linearen Transformation $v_i = \alpha_i'' + w_i(\beta_i''-\alpha_i'')$ zu

\begin{equation}
  F(\mathbf{\alpha},\mathbf{\beta}) = (\alpha_1'' \beta_1'') \int_{0}^{1} \text{ ...}(\alpha_n'' \beta_n'') \int_{0}^{1} g(\mathbf{L}[\Phi^{-1}(v_1),...,\Phi^{-1}(v_n)] + \mathbf{\mu}) d\mathbf{w}
\end{equation}

, wobei $w_i$ uniform in $[0,1]$ verteilt ist.

Wir wollen bemerken, dass $F$ als die Erwartung der Funktion $g(\mathbf{y(w)})$ interpretiert werden kann. Es ist klar, dass diese Erwartung invariant unter Permutation der Indizes
von $\mathbf{w}$ ist. Diese Darstellung erlaubt die direkte Anwendung von mehrdimensionalen Integrationsmethoden.
Auch ist an Gleichung (1.12) zu erkennen, dass sich das Integrationsproblem um eine Dimension
reduziert hat, da die rechte Seite der Gleichung nicht von $w_n$ abhängt und diese Variable daher
herausintegriert werden kann.
 

\subsection{Update Formeln für Zwei-Team-Spiele}

In diesem Kapitel werden wir die Update Formeln für Zwei-Team-Spiele. Wie im Kapitel () beschrieben, benötigen wir
zunächst die Normalisierungskonstante $Z_y$ als Funktion in Abhängigkeit von $\mathbf{\mu} \text{ und } \mathbf{\varSigma}$ für 
die drei möglichen Spielausgänge $y \in \{ -1,0, +1 \}$ (Niederlage, Unentschieden, Sieg). \\
Wir gehen davon aus, dass die Leistungen $\mathbf{x}$ der Spieler gaußverteilt sind, $\mathbf{x} \sim \mathcal{N}(\mathbf{x;s, \beta^2 I})$.
Der Likelyhood $P(y|s)$ kann dann als Ausdruck in Abhängigkeit von $z = \mathbf{a^T x}$ geschrieben werden. Aufgrund der Eigenschaften 
der Normalverteilung ist dieser ebenfalls normalverteilt mit $z \sim \mathbf{N}(z;\mathbf{a^T s, \beta^2 a^T a})$: 

\begin{equation}
 P(y|\mathbf{s}) = \begin{cases}
                     \Phi(- \epsilon; \mathbf{a^Ts, \beta^2 a^T a}), & \mbox{if } y = -1 \\
                     \Phi(\epsilon; \mathbf{a^Ts, \beta^2 a^T a}), & \mbox{if } y = 0 \\
                     1- \Phi(\epsilon; \mathbf{a^Ts, \beta^2 a^T a}), & \mbox{if } y = +1 \\
                   \end{cases}
\end{equation}

Der Likelyhood ist eine lineare Funktion von $\Theta$, ausgewertet entweder an $\epsilon \text{ oder } - \epsilon$. Es gilt

\begin{equation}
\begin{split}
 \left < \Phi \left ( \epsilon; \mathbf{a^T s, \beta^2 a^T a} \right )\right > &= \int_{- \infty}^{\infty} \int_{-\infty}^{\epsilon} \mathcal{N}(z; \mathbf{a^T s , \beta^2 a^T a}) \mathcal{N}(\mathbf{s; \mu, \varSigma})dz d \mathbf{s} \\
 &= \int_{- \infty}^{\epsilon} \left [ \int_{-\infty}^{\infty} \mathcal{N}(z; \mathbf{a^T s , \beta^2 a^T a}) \mathcal{N}(\mathbf{s; \mu, \varSigma}) d \mathbf{s} \right ] dz \\
 &= \int_{- \infty}^{\epsilon} \mathbf{N} \left ( z; \mathbf{a^T \mu, a^T(\beta^2 I + \varSigma) a} \right ) dz \\
 &= \Phi \left( \frac{\epsilon - \mathbf{a^T \mu}}{ \sqrt{\mathbf{a^T(\beta^2 I + \varSigma) a}} } \right).
 \end{split}
 \end{equation}

Damit erhalten wir für die Normalisierungskonstante $Z_y = P(y) = \int P(y| \mathbf{s}) P(\mathbf{s}) d \mathbf{s}$ des Likelyhoods in ()

\begin{equation}
 Z_y(\mathbf{\mu, \varSigma}) = \begin{cases}
                                 \Phi( \frac{- \epsilon + y \mathbf{a^T \mu}}{\sqrt{\mathbf{a^T (\beta^2 I + \varSigma) a}}} ) & \mbox{if } y \in \{ -1, +1 \} \\
                                 \Phi( \frac{ \epsilon - y \mathbf{a^T \mu}}{\sqrt{\mathbf{a^T (\beta^2 I + \varSigma) a}}} ) - \Phi( \frac{- \epsilon - y \mathbf{a^T \mu}}{\sqrt{\mathbf{a^T (\beta^2 I + \varSigma) a}}}) & \mbox{ if } y = 0 \\
                                \end{cases}
\end{equation}

Dazu haben wir die Eigenschaft $\Phi(t) = 1 - \Phi(-t)$ benutzt. Es wird sich als nützlich erweisen die Variablen 

\begin{equation}
 u_y(\mathbf{\mu}) := y \cdot \mathbf{a^T \mu}, \text{		} c(\mathbf{\varSigma}) := \mathbf{a^T} (\beta^2 \mathbf{I} + \mathbf{\varSigma}) \mathbf{a}.
\end{equation}

einzuführen. 

Wir im Folgenden die beiden Spielausgänge $y \in \{ -1,1 \}$ zusammen und dann den Fall $y = 0$ einzeln betrachten. Wir nehmen außerdem an, dass die Kovarianz der Prior und Posterior Verteilung
Diagonalmatrix sind.  \\

\textbf{Niederlage oder Gewinn } \\
Wir benutzen die in () eingeführten Kurznotationen und sehen, dass 

\begin{equation}
 Z_y(\mathbf{\mu, \varSigma}) = \Phi \left( \frac{u_y(\mathbf{\mu}- \epsilon)}{\sqrt{c (\mathbf{\varSigma})}} \right ).
\end{equation}

Entwickeln wir die Ableitungen des Logarithmus dieser Funktion nach $\mathbf{\mu}$ und $\mathbf{\varSigma}$ so erhalten wir 

\begin{equation}
 \begin{split}
  g_i &= \frac{y \cdot a_i}{\sqrt{c(\mathbf{\varSigma})}} \cdot v \left ( \frac{u_y(\mathbf{\mu})}{\sqrt{c(\mathbf{\varSigma})}}, \frac{\epsilon}{\sqrt{c(\mathbf{\varSigma})}}  \right )  , \\
  G_{i,j} &= - \frac{a_i a_j}{2 c(\varSigma)} \cdot \left ( \frac{u_y(\mathbf{\mu})}{\sqrt{c(\mathbf{\varSigma})}}, \frac{\epsilon}{\sqrt{c(\mathbf{\varSigma})}}  \right ) \cdot \frac{u_y(\mathbf{\mu}) - \epsilon}{\sqrt{c(\mathbf{\varSigma})}}.
 \end{split}
\end{equation}

Nutzen wir aus, dass $\mathbf{\varSigma} = diag(\mathbf{\sigma^2})$, so erhalten wir für die aktualisierte Erwartung und die aktualisierte Varianz

\begin{equation}
\begin{split}
 \hat \mu_i &= \mu_i + y \cdot v \frac{a_i \sigma_i^2}{\sqrt{c(\mathbf{\varSigma})}} \cdot v \left ( \frac{u_y(\mathbf{\mu})}{\sqrt{c(\mathbf{\varSigma})}}, \frac{\epsilon}{\sqrt{c(\mathbf{\varSigma})}}  \right ) \\
 \hat \sigma_i^2 &= \sigma_i^2 \cdot \left ( 1 - \frac{a_i^2 \sigma_i^2}{c(\mathbf{\varSigma})} \cdot w \left ( \frac{u_y(\mathbf{\mu})}{\sqrt{c(\mathbf{\varSigma})}}, \frac{\epsilon}{\sqrt{c(\mathbf{\varSigma})}}  \right )  \right ) \\
 \end{split}
 \end{equation}

\textbf{Unentschieden	} \\
Wir benutzen wiederum die Kurznotationen aus (). Damit ergibt sich für $Z_0(\mathbf{\mu}, \mathbf{\varSigma})$

\begin{equation}
 Z_0(\mathbf{\mu}, \mathbf{\varSigma}) = 1 - \Phi \left( \frac{u_{-1}(\mathbf{\mu}- \epsilon)}{\sqrt{c (\mathbf{\varSigma})}} \right ) - \Phi \left( \frac{u_{+1}(\mathbf{\mu}- \epsilon)}{\sqrt{c (\mathbf{\varSigma})}} \right ).
\end{equation}

Entwickeln wir die Ableitungen des Logarithmus dieser Funktion nach $\mathbf{\mu}$ und $\mathbf{\varSigma}$ so erhalten wir 

\begin{equation}
 \begin{split}
   g_i &= - \frac{a_i}{c(\mathbf{\varSigma})} \cdot \frac{ \mathcal{N}\left( \frac{u_{-1}(\mathbf{\mu}- \epsilon)}{\sqrt{c (\mathbf{\varSigma})}} \right ) - \mathcal{N}\left( \frac{u_{+1}(\mathbf{\mu}- \epsilon)}{\sqrt{c (\mathbf{\varSigma})}} \right ) }{1 - \Phi \left( \frac{u_{-1}(\mathbf{\mu}- \epsilon)}{\sqrt{c (\mathbf{\varSigma})}} \right ) - \Phi \left( \frac{u_{+1}(\mathbf{\mu}- \epsilon)}{\sqrt{c (\mathbf{\varSigma})}} \right )}, \\  
   G_{i,j} &= \frac{a_i a_j}{2c \mathbf{\varSigma}} \cdot \frac{ \mathcal{N}\left( \frac{u_{-1}(\mathbf{\mu}- \epsilon)}{\sqrt{c (\mathbf{\varSigma})}} \right ) \cdot \left( \frac{u_{-1}(\mathbf{\mu}- \epsilon)}{\sqrt{c (\mathbf{\varSigma})}} \right ) - \mathcal{N}\left( \frac{u_{+1}(\mathbf{\mu}- \epsilon)}{\sqrt{c (\mathbf{\varSigma})}} \right ) \cdot \left( \frac{u_{+1}(\mathbf{\mu}- \epsilon)}{\sqrt{c (\mathbf{\varSigma})}} \right )}{1 - \Phi \left( \frac{u_{-1}(\mathbf{\mu}- \epsilon)}{\sqrt{c (\mathbf{\varSigma})}} \right ) - \Phi \left( \frac{u_{+1}(\mathbf{\mu}- \epsilon)}{\sqrt{c (\mathbf{\varSigma})}} \right )}, \\ 
 \end{split}
\end{equation}

Wir führen zwei Funktionen $\tilde v(t, \epsilon) := v(t, - \epsilon, \epsilon), \text{ 	} \tilde w(t, \epsilon) := w(t, - \epsilon, \epsilon) .$
 
Nutzen wir aus, dass $\mathbf{\varSigma} = diag(\mathbf{\sigma^2})$ und die Symmetrie der Funktionen $\tilde v$ und $\tilde w$, so erhalten wir für die aktualisierte Erwartung und die aktualisierte Varianz
 
\begin{equation}
\begin{split}
 \hat \mu_i &= \mu_i + \frac{a_i \sigma_i^2}{\sqrt{c(\mathbf{\varSigma})}} \cdot \tilde v \frac{a_i \sigma_i^2}{\sqrt{c(\mathbf{\varSigma})}} \cdot v \left ( \frac{u_{+1}(\mathbf{\mu})}{\sqrt{c(\mathbf{\varSigma})}}, \frac{\epsilon}{\sqrt{c(\mathbf{\varSigma})}}  \right ) \\
 \hat \sigma_i^2 &= \sigma_i^2 \cdot \left ( 1 - \frac{a_i^2 \sigma_i^2}{c(\mathbf{\varSigma})} \cdot \tilde w \left ( \frac{u_{+1}(\mathbf{\mu})}{\sqrt{c(\mathbf{\varSigma})}}, \frac{\epsilon}{\sqrt{c(\mathbf{\varSigma})}}  \right )  \right ). \\
 \end{split}
 \end{equation}
 
\subsection{Update Formeln für Mehrteam-Spiele	}
 
In diesem Kapitel entwickeln wir die Update Formeln für Mehrteam-Spiele. Wie im Kapitel () beschrieben, benötigen wir
zunächst die Normalisierungskonstante $Z_y$ als Funktion in Abhängigkeit von $\mathbf{\mu} \text{ und } \mathbf{\varSigma}$ für jeden möglichen Spielausgang
$\mathbf{r} \in \{ 1,...,n \}^k$. 

Wir gehen davon aus, dass die Leistungen $\mathbf{x}$ der Spieler gaußverteilt sind, $\mathbf{x} \sim \mathcal{N}(\mathbf{x;s, \beta^2 I})$.
Der Likelyhood $P(y|s)$ kann dann als Ausdruck in Abhängigkeit von $z = \mathbf{a^T x}$ geschrieben werden. Aufgrund der Eigenschaften 
der Normalverteilung ist dieser ebenfalls normalverteilt mit $z \sim \mathbf{N}(z;\mathbf{A^T s, \beta^2 A^T A})$. Führen wir einen weiteren Vektor $\mathbf{y}$ ein, s.d. 
$y_j = 1$, falls das $j$-te Paar Teams auf benachbarten Platzierungen liegt und $y_j = 0$, falls das $j$-te Paar Teams das Spiel auf dem gleichen Rang beendet, so erhalten wir

\begin{equation}
 P(\mathbf{r|s}) := P \left ( \bigwedge_j (z_j \in I_{y_j})  \right ) = P(\mathbf{z} \in I_{\mathbf{y}}).
\end{equation}

Dabei sei $I_0 := (- \epsilon,\epsilon), I_1 := (\epsilon,+ \infty)$ und $I_{\mathbf{y}} = I_{y_1} \times I_{y_2} \times \cdots$. Für die Normalisierungskonstante $Z_{\mathbf{r}} = P(\mathbf{r}) = \int_{- \infty}^{+ \infty} P(\mathbf{r}|\mathbf{s}) P(\mathbf{s}) d \mathbf{s}$ wissen wir

\begin{equation}
\begin{split}
 Z_{\mathbf{r}}(\mathbf{\mu}, \mathbf{\varSigma}) &= \int_{- \infty}^{\infty} \int_{I_y} \mathcal{N}(\mathbf{z;A^Ts; \beta^2 A^T A}) \mathcal{N}(\mathbf{s;\mu,\varSigma}) d \mathbf{z} d \mathbf{s} \\
 &= \int_{I_y} \left [ \int_{- \infty}^{\infty} \mathcal{N}(\mathbf{z;A^Ts; \beta^2 A^T A}) \mathcal{N}(\mathbf{s;\mu,\varSigma}) d \mathbf{z} \right ] d \mathbf{s} \\
 &= \int_{I_y} \mathcal{N}(\mathbf{z; A^T \mu; A^T (\beta^2 I + \varSigma) A}) d \mathbf{z}.
 \end{split}
 \end{equation}

Um die Notation zu verkürzen führen wir die Funktionen 

\begin{equation}
 \mathbf{u(y) := A^T \mu} \text{ und 	} \mathbf{C(\varSigma) := A^T (\beta^2 I + \varSigma) A}
\end{equation}

ein. \\ 

\textbf{Update Formeln für den Erwartungswert} \\

Für das GDF Update für $\mathbf{\mu}$ müssen wir $\mathbf{g} := \frac{ \partial \log{Z_r(\mathbf{\mu,\varSigma})}}{\partial \mathbf{\mu}}$ berechnen. Mit den Ergebnissen aus 
dem vorherigen Kapitel erhalten wir:

\begin{equation}
 \begin{split}
 \mathbf{g} &= \frac{1}{Z_r(\mathbf{\mu, \varSigma})} \int_{I_y} \frac{\partial \mathcal{N}(\mathbf{z;u(\mu); C(\varSigma)})}{\partial \mathbf{\mu}} d \mathbf{z} \\
 &= \int_{I_y} \frac{\partial \mathcal{N}(\mathbf{z;u(\mu); C(\varSigma)})}{Z_r(\mathbf{\mu, \varSigma})} \cdot \frac{\partial - \frac{1}{2} (\mathbf{z-u(\mu)})^T \cdot \mathbf{C^{-1}(\varSigma) \cdot (\mathbf{z-u(\mu)})}}{\partial \mu} d \mathbf{z} \\
 &= \int_{I_y} \frac{\partial \mathcal{N}(\mathbf{z;u(\mu); C(\varSigma)})}{Z_r(\mathbf{\mu, \varSigma})} \cdot (\mathbf{A \cdot C^{-1}(\varSigma) \cdot z - A \cdot C^{-1}(\varSigma) \cdot u(\mu)}) d \mathbf{z}. \\
 \end{split}
\end{equation}

Da $I_y$ einen Hyperwürfel darstellt, können wir den Ausdruck verkürzt schreiben: 

\begin{equation}
\begin{split}
 \mathbf{g} &= \left < A \cdot C^{-1}(\varSigma) \cdot z \right >_{z \sim \mathcal{R}(\mathbf{z;u(\mu), C(\varSigma), I_y})} - \mathbf{A \cdot C^{-1}(\varSigma) \cdot u(\mu)} \\
 &= \mathbf{A \cdot C^{-1}(\varSigma) \cdot \left( \left <z \right >_{z \sim \mathcal{R}(\mathbf{z;u(\mu), C(\varSigma), I_y})} - \mathbf{u(\mu)} \right )} d \mathbf{z}.
\end{split}
 \end{equation}

 Wenden wir auf diesen Ausdruck das GDF Update an und nutzen wir die Diagonalität der Kovarianzmatrix $\mathbf{\varSigma}$ aus, so ergibt sich für die aktualisierte Erwartung: 
 
 \begin{equation}
  \hat \mu_i = \mu_i + \sigma_i^2 \mathbf{e_i^T A \cdot C^{-1}(\varSigma) \cdot \left ( \left < z \right >_{z \sim \mathcal{R}(\mathbf{z;u(\mu), C(\varSigma), I_y})} - u(\mu)  \right )}
 \end{equation}

 Im Fall von $k = 2$ Teams reduzieren sich diese Gleichungen zu den Gleichungen im vorherigen Kapitel für $y \in \{ 0,1 \}$, da $\mathbf{C(\varSigma) = c(\varSigma) \text{ und } u(\mu)} = u(\mu)$. 
 Es gilt dann 
 
 \begin{equation}
  \mathbf{\left < z \right >_{z \sim \mathcal{R}(z)} = u(\mu) + } \sqrt{c(\varSigma)} \left [ v \left( \frac{u(\mu)}{\sqrt{c(\varSigma)}}, \frac{\epsilon}{\sqrt{c(\varSigma)}} \right )^y \cdot \tilde v \left ( \frac{u(\mu)}{\sqrt{c(\varSigma)}}, \frac{\epsilon}{\sqrt{c(\varSigma)}} \right )^{1-y} \right ] .
 \end{equation}

 \textbf{ Varianz Update Formeln	} \\
 Für die GDF Update Formeln für $\varSigma$ müssen wir $\mathbf{\frac{ \partial \log{Z_r(\mu,\varSigma)}}{\partial \varSigma}}$ berehcnen. Dazu bemerken wir, dass 
  
 \begin{equation}
  \frac{\partial \mathbf{C}}{\partial \varSigma_{i,j}} = \frac{\partial \mathbf{A^T \varSigma A}}{\partial \varSigma_{i,j}} = \mathbf{A^T \left ( \frac{\partial \varSigma}{\partial \varSigma_{i,j}} \right ) A = A^T e_i e_j^T A }
 \end{equation}

 Mittels Standardregeln aus der linearen Algebra lässt sich jetzt $\mathbf{G_{i,j}}$ herleiten:
 
 \begin{equation}
  \begin{split}
   G_{i,j} &= \frac{1}{Z_r} \int_{I_y} \frac{\partial \mathcal{N}(\mathbf{z;u; C})}{\partial{\varSigma_{i,j}}} d \mathbf{z} \\
   &= \frac{1}{Z_r} \int_{I_y}  \frac{ \partial (2 \pi)^{- \frac{n}{2}} \mathbf{|C|}^{\frac{1}{2}} exp \left (- \frac{1}{2} (\mathbf{z - \mu})^T \mathbf{C^{-1}} (\mathbf{z - \mu}) \right )}{\partial \varSigma_{i,j}} d \mathbf{z} \\
   &= \frac{1}{Z_r} \int_{I_y} tr \left( \left ( \frac{ \partial (2 \pi)^{- \frac{n}{2}} \mathbf{|C|}^{\frac{1}{2}} exp \left (- \frac{1}{2} (\mathbf{z - \mu})^T \mathbf{C^{-1}} (\mathbf{z - \mu}) \right )}{\partial \mathbf{C}} \right )^T \cdot \frac{\partial \mathbf{C}}{\partial \varSigma_{i,j}} \right) d \mathbf{z} \\
   &= \frac{1}{2} \mathbf{e_j^T A C^{-1}} \left ( \left [ \int_{I_y} \frac{\mathcal{N}(\mathbf{z;u,C})}{Z_r} \mathbf{(z - u)(z-u)^T d z}- \mathbf{C} \right ] \right ) \mathbf{C^{-1} A^T e_i} \\ 
  \end{split}
 \end{equation}

 Dieser Ausdruck mithilfe der Verteilung $\mathcal{R}(\mathbf{z;u,C; I_y})$ kompakter geschrieben werden:
 
 \begin{equation}
  G_{i,j} = \frac{1}{2} \mathbf{e_j^T A C^{-1}} \left ( \left < \mathbf{z z ^T} \right >_{z \sim \mathcal{R}(z)} - \left < \mathbf{z} \right >_{z \sim \mathcal{R}(z)} \mathbf{u^T}  \mathbf{u} \left < \mathbf{z^T} \right >_{z \sim \mathcal{R}(z)} + \mathbf{u u^T - C} \right ) \mathbf{C^{-1} A^T e_i} \\
 \end{equation}

 Für die gesamte Update Formel der Varianz benötigen wir $g_i^2 - 2 G_{i,i}$: 
 
 \begin{equation}
  g_i^2 - 2 G_{i,i} = \mathbf{e_i^T A C^{-1} (C - (\left < z z^T \right >_{z \sim \mathcal{R}(z)} - \left < z \right >_{z \sim \mathcal{R}(z)} \left < z^T \right >_{z \sim \mathcal{R}(z)} )) C^{-1} A^T e_i}.
 \end{equation}

 Setzen wir diese Ergebnisse rückwärts in der GDF Update Formel ein und nutzen wir wiederum die Diagonalstruktur der Kovarianzmatrix aus, so ergibt sich für die aktualisierte Varianz:
 
 \begin{equation}
  \hat \sigma_i^2 = \sigma_i^2 (1- \sigma_i^2 \cdot \mathbf{e_i^T A C^{-1}(C- (\left < z z^T \right >_{z \sim \mathcal{R}(z)} - \left < z \right >_{z \sim \mathcal{R}(z)} \left < z^T \right >_{z \sim \mathcal{R}(z)}))} \mathbf{C^{-1} A^T e_i}).
 \end{equation}
 
 Auch hier ergeben sich die Zwei-Team-Spiele Update Formeln als Spezialfall aus den hier hergeleiteten Mehrteam-Spiel Formeln: 
 
 \begin{equation}
 \begin{split}
  & \mathbf{\left < z z^T \right >_{z \sim \mathcal{R}(z)} - \left < z \right >_{z \sim \mathcal{R}(z)} \left < z^T \right >_{z \sim \mathcal{R}(z)} }   \\ 
  = &\mathbf{C(\varSigma) \cdot } \left ( 1-  \left [ w \left( \frac{u(\mu)}{\sqrt{c(\varSigma)}}, \frac{\epsilon}{\sqrt{c(\varSigma)}} \right )^y \cdot \tilde w \left ( \frac{u(\mu)}{\sqrt{c(\varSigma)}}, \frac{\epsilon}{\sqrt{c(\varSigma)}} \right )^{1-y} \right ] \right ) .
 \end{split}
 \end{equation}


 \end{document}
